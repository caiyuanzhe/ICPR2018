\section{Introduction}

Speaker Verification (SV) is the process of accepting or rejecting a person from given personal utterance. Most of the applicable services in which banking system, financial business and security control use of voice to confirm the identity of a speaker for verification purpose. For example, Ping An insurance department utilizes speaker verification technique within phone conversation between customers and client services to authenticate the caller's identification.

Telephone Customer Service (TCS) is an essential platform of customer-oriented business enterprise. It helps customers to solve urgent needs and get personalized services. In Ping An Co.,Ltd, client services are able to recommend credit card, deposit plan, insurance plan to the customer based on their history records through this platform. Within TCS, customer identification is sometimes required when there is a need to provide precision marketing, i.e. apply automobile insurance compensation. Hence, it would be beneficial if Ping An telephone customer service can apply speaker verification technique within phone call conversation to reduce the human costs and time costs.

The current system in Ping An speaker verification uses GMM-based i-vector framework. Although it has achieved great success in some real-world applications (e.g., Happy Ping An APP \footnote{\url{http://tech.Pingan.com/product/sass_b_happy.shtml}}, Ping An Zhi Bird APP \footnote{\url{http://www.zhi-niao.com/}},) this technique has not been widely used in Ping An TCS (i.e., 95511) platform.

There are three main problems in our current SV model:

\begin{itemize}
\item Co-channel speech: the current Ping An telephone service platform only records co-channel speech (single-channel, two-speaker), which cannot be directly used in SV system. Therefore, how to extract the clean and long customer voice segments is a vital problem.

\item Ping An TCS requires a very short response time (less than 1s) for speaker verification. The quicker to identify the user's information, the better services Ping An service can provide to customer.

\item The high accuracy SV algorithm is always expected in Ping An TCS.

\end{itemize}

To solve the above problems, we developed a SV system with customer/service identification front-end to extract customer speech segments from the co-channel audio data. Specifically, the system first uses automatic speech recognition (ASR) technique to transcribe mix-audio to text, voice activity detection (VAD) is implemented to segment audio. Then, text classification approach is applied to assign the customer label to these text. In the end, according to the text content and related time stamps (from ASR result,) the identification front-end can easily concatenate these customer's homogenous audio segments together. We also purposed a new DNN-UBM SV framework (we call it HLSTM-UBM) to optimize the SV accuracy with reasonable system response time.


\noindent \textbf{Contributions:} The contributions of this paper are:
\begin{itemize}

\item In this paper, we discuss the co-channel problem for current telephone customer service. We argue that why current single channel SV approach cannot be directly used in TCS speaker verification task.

\item We present a novel SV framework for this co-channel TCS. Apart from speaker's voice biometrics, the content of speaker's dialogue has also been used for identification. The long and clean customer's voice can be extracted from the the co-channel TCS. We also analyze the characters of customer service's dialogs. More accurate, efficient and scalable short text classification method has been applied in our framework.

\item Extensive experimental analysis is performed on multiple, diverse data sets to show how the proposed framework can provide more accurate and efficient results than other approaches. Meanwhile, this SV framework has been applied in Ping An TCS.

\end{itemize}

\noindent On the following sections we discuss motivations and core ideas of our single-channel two-speaker telephone customer service speaker verification system. Related works are presented in Section \ref{sec:relatedwork}. In Section \ref{sec:algorithm} we introduce the wav-to-text ASR module, the customer/service identification module for speaker diarization and the SV module for speaker verification. We summarize the experiment setup and report experiment results in \ref{sec:experiment} and conclusions are in Section \ref{sec:conclusion}.

